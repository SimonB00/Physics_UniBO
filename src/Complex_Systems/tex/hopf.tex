Sometimes the orbits of a dynamical system converge not in a single point but in limit circle near the stable point.
This is called \emph{Hopf Bifurcation}.
That type of phenomena is recurrent in nature: one example may be the Ising's model in which the system passes from a one stable point state to a two stable points state with a phase transition.
In that specific case we observe a bifurcation of the system's free energy.\\
We have to see the Hopf bifurcation as a new dynamical state that has a periodic orbit as a solution.
However, it's important to observe that a Hopf bifurcation is not an equilibrium state for the system.\\
For example, consider a circular network of identical neurons characterized by a function $\dot{x}_k$.
The dynamic is described by a simple law
$$
    \dot{x}_k=F(x_k)+\epsilon x_{k-1}
$$
in which the function F() is taken from the Fitzhug-Nagumo model and the other term represents the system dynamic.\\
If $x_P$ is an equilibrium point for all neurons then we can verify that exists an $\epsilon=\epsilon_C$ that produces a non-trivial solution.
In particular, with that value of $\epsilon$ we can create a stationary periodic state with a periodic signal (moving wave): that's a self-solution for the system.
Obviously, this example do not conserve the total energy of the system.
