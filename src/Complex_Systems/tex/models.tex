The most simple model possible is the linear model
$$
	\dot{x} = Ax
$$
Albeit this model is easy, there are a few complications that have to be taken into account:
\begin{itemize}
	\item When the number of dimensions increases it becomes complicated. 
	\item If we consider the determinant
$$
	det(\la I - A) = 0
$$
we obtain a polinomial equation, that can be very hard to solve. 
	\item To further complicate things, $A$ might not be known, but we could have an ensemble of matrices. 
	\item The point $x=0$ is always a critical point, so we can always linearize the system around this point, but if $A$ has critical points then we get those as well. 
\end{itemize}
By solving the determinant equation, we get the eigenvalues $\la$, whose study can tell a lot about the behaviour of the solutions. \\
If $Re \la \leq 0$, this means that the exponential term of the solution shrinks, so it converges. \\ \\
Robustness means that if we perturb the system, the solutions don't change too much. A model must be robust, otherwise it can fit any kind of data simply by slightly changing the parameters. \\ \\
The formal way to write the solution of such a linear system is
$$
	x(t) = x_0\exp(At)
$$
For coefficients $\la^*$ such that $Re\la^* > 0$, we have that
$$
	|\delta x| \approx |e^{\la^* t}||\delta x_0|
$$
This is very tipical, and from this rises the chaos theory. In this case, even a small error in the initial condition will increase the error in the model exponentially fast. \\ \\
We get another interesting model by adding some noise to the linear model
$$
	\dot{x} = Ax + \xi(t)
$$
We consider a special solution of the form
$$
	x = e^{At}y
$$		
and we substitute in the equation
$$
	\dot{x} = A\exp(At)y + \xi(t) = A\exp(At)y + \exp(At)\dot{y}
$$	
$$
	\dot(y) = \exp(-At)\xi(t)
$$
So the complete solution of the problem is
\begin{equation}
	x(t) = \exp(At)x_0 + \int_0^t \exp(A(t-s))\xi(s)ds
\end{equation}
If the system is stable (negative real part of lambda) and we use a periodic forcing, the solution is still a periodic function with different period. \\ \\
Now we want to consider the case of a perturbed matrix:
$$
	\dot{x} = (A+\varepsilon B)x
$$
with $\varepsilon \ll 1$. \\
The solution is
$$
	x(t) = \exp((A+\varepsilon B) t)x_0
$$
and to understand the system's sensitivity we calculate the derivative for small perturbations
$$
	\frac{d}{d\varepsilon}\exp((A+\varepsilon B)t)
$$
but the problem is that usually the two matrices are not commutative. \\
If we take the special solution $x = \exp(At)x_0$ and we substitute we get
$$
	\dot{y} = \varepsilon\exp(-At)B\exp(At)y
$$
and, if $A$ and $B$ do not commute, this equation is very difficult to solve. \\
An approximate solution is
$$
	y(t) = y_0 + \varepsilon\int_0^t \exp(-As)B\exp(As)y_0ds + O(\varepsilon^2)
$$
and the solution for x is
\begin{equation}
	x(t) = \exp(At)x_0 + \varepsilon\int_0^t \exp(A(t-s))B\exp(As)x_0ds + O(\varepsilon^2)
\end{equation}
Now, the sensitivity of the system is
$$
	\frac{dx}{d\varepsilon} = \int_0^t \exp(A(t-s))B\exp(As)x_0ds 
$$
In the particular case where the matrix is diagonal, the sensitivity becomes
$$
	\frac{dx}{d\varepsilon} = \int_0^t \exp(\la_i(t-s))B_{ij}\exp(\la_js)x_0ds 
$$
