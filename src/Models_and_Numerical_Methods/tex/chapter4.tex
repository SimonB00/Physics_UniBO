\chapter{Ergodic Theorem, AEP Theorem and Entropy Theorem}
In this chapter we want to discuss the most important theorems of Information Theory: the \textit{Ergodic theorem}, the \textit{Asymptotic Equipartition Property (AEP) theorem} and the \textit{Shannon-McMillan-Breiman theorem} or \textit{Entropy theorem}.
\\First we will need to recall some concepts in probability convergence and state some fundamental results.

\section{Some fundamental results}
We say that a sequence of random variables $x_1, x_2, \dots, $ converges to a random variable $x$:
\begin{itemize}
    \item[a)] \textbf{in probability} if $\mathbb{P}(|x_n - x| > 0) \rightarrow 0$ when $n \rightarrow \infty$ and for all $\epsilon >0$
    \item[b)] \textbf{in mean square} if $\mathbb{E}( (x_n - x)^2 ) \rightarrow 0$ when $n \rightarrow \infty$
    \item[c)] \textbf{almost surely} (also called \textbf{with probability $\bm{1}$}) if 
    \begin{equation}
        \mathbb{P} \big( \lim_{n \rightarrow \infty} x_n = x \big) = 1
    \end{equation}
\end{itemize}
Of course $c) \Rightarrow b) \Rightarrow a)$. \\
If $\{P_n\}$ is a sequence of measurable properties (for example: $\frac{\sum_i^n x_i}{n} - \Bar{x}$ $<$ smaller or bigger than $\epsilon$) then
\begin{itemize}
    \item[1)] $P_n(x)$ holds \textbf{eventually almost surely}, if for almost every $x$ there is an $N=N(x)$ such that $P_n(x)$ is true for every $n \geq N$
    \item[2)] $P_n(x)$ holds \textbf{infinitely often}, almost surely, if for almost every $x$ there is an increasing sequence $\{n_j \}$ of integers (which may depend on $x$) such that $P_{n_j}(x)$ is true for $j = 1,2, \dots$
\end{itemize}
Here, $1) \Rightarrow 2)$. We also have three equivalent formulations of almost sure convergence:
\begin{lemma}
    The following are equivalent for measurable functions on a probability space:
    \begin{itemize}
        \item[1:] $f_n \rightarrow f$, almost surely;
        \item[2:] $|f_n(x) - f(x)| < \epsilon$, eventually almost surely, for every $\epsilon>0$;
        \item[3:] Given $\epsilon>0$, there is an $N$ and a set $G$ of measure at least $1 - \epsilon$ such that $|f_n(x) - f(x)| < \epsilon$, $x \in G$, $n \geq N$.
    \end{itemize}
\end{lemma}
\begin{lemma}[The Markov inequality]
\label{lemma:markov_ineq}
    Let $f$ be a nonnegative, integrable function on a probability space $(X, \Sigma, \mu)$. 
    \\If 
    \begin{equation*}
        \int f \, d \mu \leq \epsilon \delta
    \end{equation*}
    then 
    \begin{equation*}
        f(x) \leq \epsilon
    \end{equation*}
    except for a set of measure at most $\delta$. 
\end{lemma}
Next, we state a very deep and important result, the Borel-Cantelli principle \cite{Borel_cantelli}: such Lemma states that, under certain conditions, an event will have probability of either zero or one. Accordingly, it is the best-known of a class of similar theorems, known as zero-one laws.
\begin{lemma}[The Borel-Cantelli principle]
    If $\{E_n \}$ is s a sequence of measurable sets (so, a series of events) in a probability space $(X, \Sigma, \mu)$ such that
    \begin{equation*}
        \sum_n^{\infty} \mu(E_n) < \infty
    \end{equation*}
    then 
    \begin{equation*}
        x \notin E_n 
    \end{equation*}
    eventually almost surely, or equivalently 
    \begin{equation*}
        \mu \big( \limsup_{n \rightarrow \infty} E_n \big) = 0
    \end{equation*}
\end{lemma}
Here, "lim sup" denotes limit supremum of the sequence of events, and each event is a set of outcomes. That is, $\limsup E_n$ is the set of outcomes that occur infinitely many times within the infinite sequence of events $\{ E_n \}$. Explicitly,
\begin{equation*}
    \limsup_{n \rightarrow \infty} E_n = \cap_{n=1}^\infty \cup_{k=n}^\infty E_k
\end{equation*}
The set $\limsup E_n$ is sometimes denoted $\{E_n \, \text{i.o.} \}$, where "i.o." stands for "infinitely often". The theorem therefore asserts that if the sum of the probabilities of the events $E_n$ is finite, then the set of all outcomes that are "repeated" infinitely many times must occur with probability zero. Note that no assumption of independence is required.
\\This basically means that such $E_n$ are becoming smaller and smaller with increasing $n$: thus if for example the sets $E_n$ are "bad" sets in which we do not want our variable $x$ to be, we just need to prove that $\sum_n \mu(E_n) < \infty$ to "neglect them" in the long run. 
\\A related result, sometimes called the \textbf{second Borel–Cantelli lemma}, is a partial converse of the first Borel–Cantelli lemma. The lemma states: If the events $E_n$ are independent and the sum of the probabilities of the $E_n$ diverges to infinity, then the probability that infinitely many of them occur is $1$. That is:
\begin{lemma}[second Borel–Cantelli lemma]
    if $\sum_{n=1}^\infty \mu(E_n) = \infty$ and the events $\{ E_n \}_{n=1}^\infty$ are \textbf{pairwise independent}, then 
    \begin{equation*}
        \mu \big( \limsup_{n \rightarrow \infty} E_n \big) = 1
    \end{equation*}
\end{lemma}
The infinite monkey theorem, that monkeys endless typing at random will, with probability $1$, eventually produce every finite text (such as the works of Shakespeare) \footnote{However, the probability that monkeys filling the entire observable universe would type a single complete work, such as Shakespeare's Hamlet, is so tiny that the chance of it occurring during a period of time hundreds of thousands of orders of magnitude longer than the age of the universe is extremely low (but technically not zero). To put it another way, for a one in a trillion chance of success of the monkeys typing Hamlet, there would need to be $10^{360641}$ observable universes where \textit{each proton} was a monkey typing.}, amounts to the statement that a (not necessarily fair) coin tossed infinitely often will eventually come up Heads. This is a special case of the second Lemma.
\\To end this section, we state an important result connecting cardinality and probability:
\begin{lemma}[cardinality bounds]
    Let $\mu$ be a probability measure on the finite set $A$, let $B \subset A$, and let $\alpha$ be a positive number.
    \begin{itemize}
        \item[1)] if $a \in B \Rightarrow \mu(a) \geq \alpha$, then $|B| \leq 1/\alpha$.
        \item[2)] For $b \in B$, $\mu(b) \geq \alpha/|B|$, except for a subset of $B$ of measure at most $\alpha$
    \end{itemize}
\end{lemma}
Basically, we are saying that given a set in a measurable space, this is \textbf{either big in probability or in cardinality}. We cannot have both. 

\section{AEP property and SMB Theorem}
In information theory, the analog of the law of large numbers is the asymptotic equipartition property (AEP). It is a direct consequence of the weak law of large numbers.
\begin{theorem}[AEP Theorem]
    \hfill \\
    If $x_1, x_2, x_3, \dots$ are i.i.d. which follow a distribution $p(x)$, then 
    \begin{equation}
        -\frac{1}{n} \log p(x_1, x_2, \dots, x_n) \xrightarrow{\text{in probability}}  H(x)
    \end{equation}
\end{theorem}
\begin{proof}
\hfill
\\
    Functions of independent random variables are also independent random variables. Thus, since the $x_i$ are i.i.d., so are $\log p(x_i)$. Hence, by the weak law of large numbers, 
    \begin{equation*}
        -\frac{1}{n} \log p(x_1, x_2, \dots, x_n) = -\frac{1}{n} \sum_i \log p(x_i) \xrightarrow{\text{in probability}} - \mathbb{E}[\log p(x)] = H(x)
    \end{equation*}
    which proves the theorem.
\end{proof}
We can also give an alternative (and equivalent) formulation of the AEP Theorem, stated under the assumptions of considering a  discrete-time stationary ergodic process $\mu$: the asymptotic equipartition property for such stochastic source is known as the Shannon–McMillan–Breiman theorem (SMB).
\begin{theorem}[SMB Theorem]
\label{th:SMB}
    \hfill \\
    Consider a stationary, ergodic, stochastic source $\mu$ with entropy $h = h_\mu > 0$. For simplicity, here we also assume that we are dealing with an i.i.d. process (but the more general case is also true).
    \\Then for every $\epsilon>0$ and all sufficiently large $n$, one can find a subset $\mathcal{T}_n \subset \alp^n$ of \textbf{typical sequences} that is \textbf{small} in cardinality, but \textbf{large} in probability, so that:
    \begin{itemize}
        \item[(1)] $e^{n (h- \epsilon)} \leq |\mathcal{T}_n| \leq e^{n (h+ \epsilon)}$
        \item[(2)] $\lim_{n \rightarrow \infty} \mu(\mathcal{T}_n) = 1$
        \item[(3)] For each $\omega \in \mathcal{T}_n$ we have
        \begin{equation*}
            e^{n (h- \epsilon)} \leq \mu(\omega) \leq e^{n (h+ \epsilon)}
        \end{equation*}
    \end{itemize}
\end{theorem}
Let's take a closer look at what this theorem is telling us.
\\If we consider $|\alp|=m$, then $|\alp^n| = m^n = e^{n \log m}$. Also, we know from the basic properties of entropy that $0 \leq h \leq \log n$. If we then look at the ratio $\frac{|\mathcal{T}_n|}{|\alp^n|} \sim e^{-n (\log m - h)} \xrightarrow[n \rightarrow \infty]{} 0$. This is what we mean by "small" in cardinality: not in a general sense, but small with regard to the dimension of our alphabet. 
\\Also, we are saying that $\mu(\mathcal{T}_n) \rightarrow 1$ as $n \rightarrow \infty$: i.e., the "density" of $\mathcal{T}_n$ in $\alp^n$ is becoming bigger and bigger as we increase $n$. In the limit of $n$ going to infinity, the measure of our typical set becomes $1$: hence, in such limit \textbf{all sequences are typical}. 
\\All of this is just to say that under suitable conditions, we will always have in our alphabet a few sequences with are typical with respect to all of those which can be formed: think as always about words in a language versus all the combinations of letters that can be formed. 
\\Also, the last statement is telling us that $\mu(\omega) \sim e^{-n h}$ up to an $\epsilon$, which is equivalent to saying that $\lim_{n \rightarrow \infty} \frac{1}{n} \log \mu(\omega_n) = h$ as in the first formulation.
\begin{proof}
    \hfill \\ 
    Given $\epsilon>0$, we let $\delta=\delta(\epsilon)$ to be chosen later.
    \\We select  finite sequences with good empirical distribution:
    \begin{equation*}
        \mathcal{T}_n \equiv \{ \omega \in \alp^n : \big| \frac{f_\omega(a_k)}{n} - \mu_k \big| \leq \delta, \, k=1,2,\dots, d \}
    \end{equation*}
    The second statement follows immediately from the Law of Large Numbers: $\lim_{n \rightarrow \infty} \mu(\mathcal{T}_n) = 1$.
    \\In the i.i.d. case we are discussing here:
    \begin{align*}
        \mu(\omega) = & \prod_{j=1}^n \mu(\omega_j) = \prod_{k=1}^d \mu_k^{f_\omega(a_k)} =  \\
        = & \exp \big( \sum_{k=1}^d f_\omega(a_k) \ln \mu_k \big) = \exp \big( n \sum_{k=1}^d \frac{f_\omega(a_k)}{n} \ln \mu_k \big) = & \\
        = & \exp \big( n \sum_{k=1}^d \mu_k \ln \mu_k \big) \cdot \exp \big( n \sum_{k=1}^d (\frac{f_\omega(a_k)}{n} - \mu_k) \ln \mu_k \big) = & \\
        = & \exp \bigg( n (-h + \sum_{k=1}^d (\frac{f_\omega(a_k)}{n} - \mu_k) \ln \mu_k) \bigg)
    \end{align*}
    Now we choose $\delta$ sufficiently small such that $\big|  \sum_{k=1}^d (\frac{f_\omega(a_k)}{n} - \mu_k) \ln \mu_k) \big| \leq \epsilon$, for $\omega \in \alp^n$, and this gives the first statement of the Theorem.
    \\Furthermore, 
    \begin{equation*}
        1 \geq \mu(\mathcal{T}_n) = \sum_{\omega \in \alp^n} \mu(\omega) \geq e^{-n (h + \epsilon)} |\mathcal{T}_n|
    \end{equation*}
    i.e., $|\mathcal{T}_n| \leq e^{n(h+ \epsilon)}$.
    \\On the other hand, for sufficiently large $n$ we will have that
    \begin{equation*}
        \frac{1}{2} \leq \mu(\mathcal{T}_n) \leq e^{-n(h - \frac{\epsilon}{2})} |\mathcal{T}_n| \, ,
    \end{equation*}
    i.e., for sufficiently large $n$
    \begin{equation*}
        |\mathcal{T}_n| \geq \frac{1}{2} e^{n(h - \frac{\epsilon}{2})} \geq e^{n(h - \epsilon)} 
    \end{equation*}
    and this gives the first statement.
\end{proof}
We now want to prove a a result much stronger than the SMB theorem: the \textbf{Entropy Theorem}

\section{Entropy Theorem and Ergodic Theorem}

\begin{theorem}[Entropy Theorem]
\hfill \\
    Given a stochastic,stationary, ergodic process $\mu$ on $\alp$,
    \begin{equation}
        \lim_{n \rightarrow \infty} -\frac{1}{n} \log \mu(x_1^n) = h_\mu \qquad \text{\textbf{almost surely}}
    \end{equation}
\end{theorem}
We want to prove this by introducing and using a beautiful technique
developed by Ornstein and Weiss, while trying to extend entropy theorem
from stochastic processes to random fields. In order to introduce the first fundamental lemmas concerning packings and covering, It is useful to start by proving the Ergodic Theorem.
\\Recalling that a process $\mu$ is ergodic if any shift-invariant subset is either of measure $1$ or $0$, we state and prove the ergodic theorem in its essential form assuming (without loss of generality) a binary process $\alp= \{0,1 \}$
\begin{theorem}[Ergodic Theorem]
\hfill \\
    If $\mu$ is a (binary) stationary, ergodic process
    \begin{equation*}
        \lim_{n \rightarrow \infty} \frac{1}{n} \sum_{j=1}^n x_j \quad \text{exists \textbf{almost surely} on } \alp^{\nb}
    \end{equation*}
\end{theorem}
In this case, letting $p = \mathbb{E}(x_1) = \mu(x_1=1)$, the theorem asserts that 
\begin{equation*}
    \lim_{n \rightarrow \infty} \frac{1}{n} (x_1 + x_2 +\dots+ x_n) = p \qquad \text{almost surely}
\end{equation*}
\begin{proof}
    We start by assuming that the Theorem is false: the superior (inferior) limit of the averages is then larger (smaller) than $p$ by a fixed amount on a set of positive measure.
    \\By assumption there exists $\epsilon>0$ such that if 
    \begin{equation*}
        A_\epsilon \equiv \big\{ x \in \alp^n : \limsup_{n \rightarrow \infty} \frac{x_1 + x_2 +\dots+ x_n}{n} > p + \epsilon \big\}
    \end{equation*}
    then $\mu(A_\epsilon) > 0$.
    \\It is then easy to see (\textit{exercise - and an important one}) that $A_\epsilon$ is \textit{invariant}, hence of measure one: 
    \begin{equation*}
        \mu(A_\epsilon) = 1
    \end{equation*}
    This means that for almost all sequences $x$, there exists an infinite partition ($x$-dependent) in disjoint intervals over which the average exceeds $p+\epsilon$.
    \\i.e. given $x= x_1, x_2, \dots $ and $n \in \nb^+$, we denote by $m=m(n) > n$ the first time
    for which the average over the interval $[n, m]$ exceeds the expected value $p$ by $\epsilon$. 
    \\$m$ is also $x$ dependent, and it is also finite for all $n$ and for a set of full measure of $x$'s in $\alp^{\nb}$.
    \\It follows that for almost all $x$, we can find a disjoint partition ($x$ dependent):
    \begin{equation*}
        \nb = \cup_{k=1}^\infty [n_k, m_k], \quad \text{with} \, \, m_0 = 0, \, n_k = m_{k-1}+1
    \end{equation*}
    and 
    \begin{equation*}
        \frac{1}{m_k - n_k + 1} \sum_{j=n_k}^{m_k} x_j > p + \epsilon \, .
    \end{equation*}
    The important step is now to obtain a control over finite sequences, proving that for sufficiently large $N$ there exists a set $\mathcal{G}_N$, large in probability, on which the average beats $p+\epsilon$.
    Let's formally state this last passage: 
    \begin{lemma}
    \label{lemma:ergodic_th}
        Given $\delta>0$ there exists a positive integer $N$ and a set $\mathcal{G}_N \subset \alp^N$ such that
        \begin{itemize}
            \item[1)] $\mu(\mathcal{G}_N)> 1 - \delta$
            \item[2)] For each $\omega_1^n = (\omega_1, \dots, \omega_n) \in \alp^N$ there are disjoint intervals $[n_i, m_i] \subset [1,N]$ such that 
            \begin{itemize}
                \item[a)] $\{ [n_i,m_i] \}$ is a $\delta-$cover of $[1,N]$, i.e. $\sum_i (m_i - n_i +1) > (1 - \delta)N$
                \item[b)] On each element of the coverings, the average exceeds $p$:
                \begin{equation*}
                    \frac{1}{|\omega_{n_i}^{m_i}|} \sum_{j=n_i}^{m_i} \omega_j > p + \epsilon
                \end{equation*}
            \end{itemize}
        \end{itemize}
    \end{lemma}
    This basic lemma leads to our desired contradiction.
    \\As a matter of fact, if $\omega_1^n \in \mathcal{G}_N$ then 
    \begin{equation*}
        \sum_{j=1}^N \omega_j \geq \sum_i \sum_{j=n_i}^{m_i} \omega_j \geq (1 - \delta) N (p + \epsilon)
    \end{equation*}
    hence
    \begin{equation*}
        p = \mathbb{E} \bigg[ \frac{1}{N} \sum_{j=1}^N X_j \bigg] \geq (1 - \delta) (p + \epsilon) \mu(\mathcal{G}_N) \geq (1 - \delta)^2 (p+\epsilon)
    \end{equation*}
    which can not be true for all $\delta$.
\end{proof}
Let's now see for completeness the proof of the Lemma:
\begin{proof}[proof of Lemma \ref{lemma:ergodic_th}]
    The first observation is that the lengths $L_n = ( m(n) -n +1 )$ of the intervals where the average is larger then $p + \epsilon$ is in fact bounded, except for a set of (arbitrary) small probability.
    \\Namely, there exists an $L$ such that $\mu(L_n > L) < \delta^2/2$.
    \\We now define $c_n$ to be $1$ if $L_n>L$ and $0$ otherwise, so that, for any $N$,
    \begin{equation*}
        \mathbb{E} \bigg[ \frac{1}{N} \sum_{n=1}^N c_n \bigg] < \frac{\delta^2}{2}
    \end{equation*}
    The Markov inequality (\ref{lemma:markov_ineq}) yields:
    \begin{equation*}
        \mu \bigg( \frac{1}{N} \sum_{n=1}^N c_n < \frac{\delta^2}{2} \bigg) > 1 - \delta
    \end{equation*}
    For $N>L$ we define $\mathcal{G}_N$ to be the event
    \begin{equation*}
        \frac{1}{N-L+1} \sum_{n=1}^{N-L+1} c_n < \frac{\delta}{2}.
    \end{equation*}
    The Markov inequality gives the first property of the basic Lemma.
    \\For each $(\omega_1, \dots, \omega_N) \in \mathcal{G}_N$ we define the intervals $[n_j, m_j]$ inductively, putting 
    \begin{equation*}
        n_1 = \min \{ n \geq 1: L_n \leq L \}, \quad m_1 = m(n_1)
    \end{equation*}
    and 
    \begin{equation*}
        n_{j+1} = \min \{ n > m_j : L_n \leq L \}, \quad m_{j+1} = m(n_{j+1}).
    \end{equation*}
    i.e.,  we go from a covering to a packing with gaps. The construction stops the first time $m_j$ exceeds $N-L+1$.
    \\The first of the second properties of the Basic Lemma is now true by definition of $n_j$ and $m_j$.
    \\An integer $n$ in $[1,N]$  fails to be in one of the intervals $[n_j, m_j]$ only if $L_n > L$ or if $n > N - L + 1$.
    \\By definition of $\mathcal{G}_N$ there are at most $N\delta/2$ indices $n$ in $[1,N]$ for which $L_n > L$.
    \\Furthermore if we choose $N \geq 2L/\delta$ then $N-L+1, N$ contains at most $N\delta/2$ integers.
\end{proof}

\section{From coverings to packings}
Now we want to discuss four Lemmas : the packing Lemma, the counting
Lemma, the doubling Lemma and the strong doubling Lemma; the first two are strictly combinatorial, while the second two are just extensions of the Ergodic Theorem.
\\The word doubling come from the fact that the applications of these
lemmas rely on the success on some other limit theorem, a
convergence-in-probability theorem for doubling and an almost-sure theorem for strong doubling. 
\\At the end we will then introduce, discuss and prove the so called strong packing lemma that will be used in the prove of the optimality of the Lempel-Ziv code.

\subsection{Packing and counting}
The packing technique is a method for building "almost" packings of intervals from "almost" covering by subintervals whose left end points already cover most of the interval. 
\\In more mathematical terms: 
\begin{definition}
    \hfill
    \\Given an interval of integers $[1, N]$ and a collection $\mathcal{C}$ of distinct subintervals, $\mathcal{C}= \{ n_j, m_j \}$ with $1 \leq n_j \leq m_j \leq N$, we say that
    \begin{itemize}
        \item[(1)] $\mathcal{C}$ is called a \textbf{strong} $\bm{1 - \delta}$- \textbf{cover} of $[1, N]$ if the left points cover at least a $1- \delta$ fraction of $[1, N]$:
        \begin{equation*}
            |\{ n_j : [n_j, m_j] \in \mathcal{C} \}| > (1 - \delta)N \,.
        \end{equation*}
        \item[(2)] $\mathcal{C}$ is called a $\bm{1 - \delta}$- \textbf{packing} if it is \textbf{disjoint} and its union cover at least $(1 - \delta)$-fraction of $[1,N]$:
        \begin{equation*}
            \sum_{[n_j, m_j] \in \mathcal{C}} |(m_j - n_j + 1)| \geq (1 - \delta)N \, .
        \end{equation*}
        \item[(3)] $\mathcal{C}$ is $\bm{L-}$ \textbf{bounded}  if $|(m_j - n_j +1 )| \leq L$, for each $[n_j, m_j] \in \mathcal{C}$
    \end{itemize}
\end{definition}
\begin{lemma}[The packing lemma]
\hfill\\
    if $N \geq L/\delta$, then any $L-$bounded, strong $1-\delta$ cover $\mathcal{C}$ of $[1,N]$ contains a $(1- 2\delta)-$packing.
\end{lemma}
The next counting lemma gives us bounds on the number of sequences that are mostly packed by block drawn from fixed collections whose size are known, which we usually consider to be the collections of typical sequences provided by the AEP theorem, or similar.
\\First we need few notations: let $\delta$ and $M$ be positive numbers: for each $m \geq M$, we have subsets $\mathcal{B}_m \subset \alp^m$ and we denote 
\begin{equation*}
    \mathcal{B}_M = \cup_{m \geq M} \mathcal{B}_m
\end{equation*}
Then we define:
\begin{definition}
    A sequence $x_1^N \in \alp^N$ is said to be $(1- \delta)-$built-up from $\mathcal{B}$ if it can be parsed in variable length blocks as 
    \begin{equation}
        x_1^N= b_1b_2\dots b_k, \quad \text{s.t.} \, \, \sum_{j: b_j \in \mathcal{B}} |b_j| \geq (1-\delta)N
    \end{equation}
\end{definition}
We denote by $G_N$ the subset of $\alp^N$ given by the sequences $x_1^N$ that are $(1-\delta)-$built-up from $\mathcal{B}_M$ .
\begin{lemma}[The Counting Lemma]
\hfill \\
    If $|\mathcal{B}_m| \geq 2^{m (h+\epsilon)}$, $h_b(2/M) \leq \epsilon/2$ and $\delta \log |\alp| \leq \epsilon/2$, then 
    \begin{equation*}
        |G_N| \leq 2^{N(h + 2\epsilon)}\, .
    \end{equation*}
\end{lemma}
where $h_b(p) = -p \log p - (1-p) \log(1-p)$ denotes the binary entropy function.
\begin{proof}
    \hfill \\
    We first condition on the locations of the blocks that belong to $\mathcal{B}_M$: a \textbf{skeleton} $\mathcal{P}= \{ [n_j, m_j] \}$ is a disjoint collection of subintervals such that
    \begin{itemize}
        \item[a)] $[n_j, m_j] \subset [1,N]$ and $ (m_j - n_j + 1) \geq M$;
        \item[b)] $\sum_{[n_j, m_j] \in \mathcal{P}} (m_j - n_j + 1) \geq (1 - \delta) N $  . 
    \end{itemize}
    We now say that $x_1^N \in G_N$ is compatible with the skeleton $\mathcal{P}$  if the word $x_i^j$ belongs to $\mathcal{B}_M$ whenever the interval $[i,j]$ belongs to $\mathcal{P}$, and we denote by $G_N(\mathcal{P}) \subset G_N$ the set of all the sequences compatible with the skeleton $\mathcal{P}$. \\
    Clearly,
    \begin{equation*}
        G_N \equiv \cup_{\mathcal{P}} G_N (\mathcal{P}) \, .
    \end{equation*}
    It is now easy to see that:
    \begin{equation*}
        |G_N (\mathcal{P})| \leq 2^{N (h+\epsilon)} |\alp|^{\delta N} \, .
    \end{equation*}
    The first term $2^{N (h+\epsilon)}$ come from the fact that any subinterval $[i, j]$ of length $m$ can be filled in at most $2^{m (h+\epsilon)} \geq |\mathcal{B}_M|$ ways, whereas there are at most $|\alp|^{\delta N}$ ways to fill the places that do not belong to $\mathcal{P}$.\\
    Each interval in a skeleton has length at least $M$, so at most $2N/M$ points can be endpoints of its intervals. Thus the number of possible skeletons is upper bounded by
    \begin{equation*}
        \sum_{j \leq 2N/M} 
        \begin{pmatrix}
            N \\
            j
        \end{pmatrix}
        \leq 2^{N h_b (2/M)}
    \end{equation*}
    The set $G_N$ is the union of the sets $G_N(\mathcal{P})$ over all skeletons $\mathcal{P}$, so the cardinality of $G_N$ is upper bounded by the product of the two previous bounds, that is
    \begin{equation*}
        \log |G_N| \leq N (h + \epsilon) + \delta N \log |\alp| + N h_b (2/M) \, .
    \end{equation*}
    This is bounded by $N (h + 2\epsilon)$ if $h_b (2/M) \leq \epsilon/2$ and $\delta \log |\alp|$ .
\end{proof}

\subsection{Doubling}
We now want to prove that in certain situations, given sequences of blocks $\mathcal{B}_n \subset \alp^n$ provided by some convergence in probability limit theorem, eventually almost surely, most indices in $x_1^N$ are in fact starting points of blocks from the $\mathcal{B}_n$'s.


